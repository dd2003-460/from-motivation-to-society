\documentclass[../main.tex]{subfiles}

\begin{document}

\chapter{V.1 存在、稳态与演化动力学}
\label{math:existence}

\mathlink{math:existence}{正文链接:存在为何需要稳态条件?}

\begin{mathbox}{1. 问题动机与建模视角}
	\textbf{问题}:在正文中,我们多次使用"存在""幸存""持续"等概念。它们在数学上意味着什么?
	
	\textbf{动机}:若一个结构不能在扰动下保持,其"存在"只是一瞬事件,而非可讨论对象。
	
	\textbf{建模视角}:将"存在"视为动力系统中的\lowfreq{稳态(Stable State)}。
\end{mathbox}

\begin{mathbox}{2. 基本模型与假设}
	\textbf{假设 A(动力系统)}:系统状态 $x(t)$ 满足
	\[ \dot{x} = f(x, \theta) \]
	其中 $\theta$ 表示环境参数。
	
	\textbf{假设 B(存在条件)}:存在 $x^*$ 使得 $f(x^*,\theta)=0$,且在小扰动下系统回归 $x^*$。
\end{mathbox}

\begin{mathbox}{3. 推论与方法论分叉}
	\textbf{推论 1(负反馈)}:若 $\partial f/\partial x <0$,系统表现为稳态,对应\lowfreq{归纳方法论}(平均、统计、拟合)。
	
	\textbf{推论 2(正反馈)}:若 $\partial f/\partial x >0$,系统远离原态,对应\lowfreq{演绎/构造方法论}(规则、制度、设计)。
	
	\textbf{未闭合分支}:多稳态与相变条件,此处不展开。
\end{mathbox}

\label{math:various}

\mathlink{main:various}{正文链接:多样性跟什么有关?}


\end{document}
