\documentclass[../main.tex]{subfiles}

\begin{document}

\chapter{V.3 环境、条件概率与历史路径}
\label{math:history}

\mathlink{math:history}{正文链接:为何历史不可重复?}

\begin{mathbox}{1. 问题动机}
	\textbf{问题}:若历史是因果结果,为何不能简单外推?
	
	\textbf{动机}:正文中大量使用"如果当初不同……"的反事实讨论,需要概率工具支撑。
\end{mathbox}

\begin{mathbox}{2. 条件概率模型}
	设事件 $H$ 为历史结果,$E$ 为环境条件,则
	\[ P(H|E) \neq P(H) \]
	
	\textbf{关键点}:改变 $E$ 并非线性改变 $H$,而是可能切换整个概率分布。
\end{mathbox}

\begin{mathbox}{3. 推论与边界}
	\textbf{推论}:历史推演只能在给定条件空间内进行。
	
	\textbf{未覆盖问题}:多条件耦合下的路径锁定。
\end{mathbox}

\end{document}
