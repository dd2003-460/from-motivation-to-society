\documentclass[../main.tex]{subfiles}

\begin{document}

\chapter{V.4 正反馈、自组织与社会结构}
\label{math:selforg}

\mathlink{math:selforg}{正文链接:经济、文化与权力的同构性}

\begin{mathbox}{1. 统一动机}
	\textbf{问题}:为何经济预期、文化认同、政治权力表现出相似的放大行为?
	
	\textbf{建模动机}:寻找跨领域的统一动力学描述。
\end{mathbox}

\begin{mathbox}{2. 抽象模型}
	设状态变量 $x$ 表示"被认同程度",其演化为
	\[ \dot{x} = ax - bx^3 \]
	其中 $a>0$ 表示正反馈强度,$b$ 表示资源/约束。
\end{mathbox}

\begin{mathbox}{3. 领域映射}
	\begin{itemize}
		\item \textbf{经济}:预期作为 $x$,价格与投资形成正反馈;
		\item \textbf{文化}:模因传播强度作为 $x$;
		\item \textbf{政治}:权力合法性作为 $x$。
	\end{itemize}
	
	\textbf{差异来源}:约束项 $b$ 的物理含义不同。
\end{mathbox}

\end{document}
