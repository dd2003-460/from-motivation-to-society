\documentclass[../main.tex]{subfiles}

\begin{document}
	
	\chapter{结语:循环的闭合}
	\begin{motivation}
		回到原点。我们构建的这个社会系统,最终又是如何反作用于"自然环境"的?
	\end{motivation}
	
	\section{人定胜天?社会系统对初始条件的修正}
	虽然地理环境为文明发展提供了初始条件和基本限制,但人类社会系统并非完全被动地接受这些条件。通过技术进步、社会组织创新等方式,人类社会反过来也在改变着自然环境。这种反作用力日益增强,甚至在某些情况下超过了自然力量本身,形成了"人化自然"的新格局。
	
	\section{新的循环:技术作为新的变量}
	技术发展为传统的地理决定论增添了新的维度。新技术不仅改变了人类与自然环境的关系,也重塑了社会内部的权力结构和组织形式。在这种情况下,社会发展的路径不再完全由原始地理条件决定,而是由技术、制度和环境三者的相互作用所塑造,形成了新的动态循环。
	
\end{document}