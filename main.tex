\documentclass[openany, a4paper, 12pt]{book}

% ==========================================
% 1. 基础宏包配置
% ==========================================
\usepackage[UTF8]{ctex}     % 中文支持
\usepackage{geometry}       % 页面设置
\geometry{left=1in, right=1in, top=1in, bottom=1in}
\usepackage{fancyhdr}       % 页眉页脚
\usepackage{hyperref}       % 超链接目录
\hypersetup{colorlinks=true, linkcolor=blue, filecolor=magenta, urlcolor=cyan}
\usepackage{subfiles}       % 子文件支持

% ==========================================
% 2. 绘图与框图工具 (修复 TikZ 和 Environment undefined 错误)
% ==========================================
\usepackage{tikz}           % 绘图核心包
\usetikzlibrary{shapes.geometric, arrows.meta, positioning} % 载入TikZ扩展库

\usepackage{tcolorbox}      % 用于制作漂亮的文本框

% ==========================================
% 3. 自定义环境定义
% ==========================================

% 定义【逻辑地图】环境
% 使用 tcolorbox 包装,带有标题和边框
\newenvironment{logicmap}{
	\begin{tcolorbox}[
		colback=blue!5!white,      % 背景色:极淡蓝
		colframe=blue!50!black,    % 边框色:深蓝
		title=\textbf{逻辑地图 (Logic Map)}, % 标题
		fonttitle=\bfseries,
		sharp corners=south,       % 底部直角
		boxrule=0.5mm
		]
	\centering
	}{
	\end{tcolorbox}
}

% 定义【写作动机】环境
% 使用灰色背景强调
\newenvironment{motivation}{
	\vspace{0.5em}
	\begin{tcolorbox}[
		colback=gray!10,           % 背景色:淡灰
		colframe=gray!60,          % 边框色:深灰
		title=\textit{本章动机 (Motivation)},
		fonttitle=\bfseries,
		arc=0mm,                   % 直角边框
		leftrule=3mm               % 左侧加粗
		]
		\small \itshape % 字体变小并倾斜
	}{
	\end{tcolorbox}
	\vspace{1em}
}

% ==========================================
% 4. 书籍信息
% ==========================================
\title{\textbf{社会动力学与公理化体系}\\ \large ——从变长编码到文明演化的统一框架}
\author{D·declaim}
\date{\today}

% ==========================================
% 5. 正文开始
% ==========================================
\begin{document}
	
	\maketitle      % 生成封面
	\tableofcontents % 生成目录
	
	\mainmatter     % 正文页码开始
	
	% =================================================================
	% 引言:动机与方法论
	% =================================================================
	\subfile{part1_core_logic/introduction}

	% =================================================================
	% PART I: 核心动机与工具
	% =================================================================
	\part{核心逻辑:自然、结构与语言 (The Core Loop)}
	\subfile{part1_core_logic/chapter1_nature_motivation}
	\subfile{part1_core_logic/chapter2_structure_efficiency}
	\subfile{part1_core_logic/chapter3_language_information}
	
	% =================================================================
	% PART II: 初始条件 - 地理环境决定论
	% =================================================================
	\part{初始输入:地理与历史的必然性 (The Geologics)}
	\subfile{part2_geological_determinism/chapter1_materialist_historical_view}
	\subfile{part2_geological_determinism/chapter2_survival_strategy_divergence}
	
	% =================================================================
	% PART III: 系统展开
	% =================================================================
	\part{系统展开:经济、文化与政治 (The Structure)}
	\subfile{part3_systematic_expansion/chapter1_economic_base}
	\subfile{part3_systematic_expansion/chapter2_culture_folk_institutions}
	\subfile{part3_systematic_expansion/chapter3_politics_power_structure}
	\subfile{part3_systematic_expansion/chapter4_law_institutional_form}
	
	% =================================================================
	% 结语与附录
	% =================================================================
	\part{结语与附录}
	\subfile{conclusion/circular_closure}
	
	\appendix
	\part{附录:数学工具箱与参考文献}
	\chapter{详细数学推导}
	\section{变长编码与熵的计算细节}
	\section{动力学方程的稳定性证明}
	
\end{document}