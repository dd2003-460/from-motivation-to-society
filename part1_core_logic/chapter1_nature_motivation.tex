\documentclass[../main.tex]{subfiles}

\begin{document}
	
	% 逻辑地图:这里使用了 tikz 绘图
	\begin{logicmap}
		\centering
		\begin{tikzpicture}[
			node distance=2.5cm,
			auto,
			block/.style={rectangle, draw=blue!60, fill=blue!5, thick, align=center, rounded corners},
			line/.style={-Latex, thick}
			]
			% 定义节点
			\node [block] (nature) {存在或者毁灭\\(生存概率)};
			\node [block, right=of nature] (struct) {结构与效率\\(负熵)};
			\node [block, right=of struct] (lang) {语言与分类\\(压缩)};

			% 定义连线
			\draw [line] (nature) -- (struct);
			\draw [line] (struct) -- (lang);
			\draw [line, dashed] (lang) to[bend left=45] node[above, font=\footnotesize] {解释与重构} (nature);
		\end{tikzpicture}

		\vspace{1em}
		\textbf{当前位置}:在探讨"结构如何形成"之前,必须先问"为什么有些东西能存在,而有些必然消亡"。
	\end{logicmap}

	\chapter{存在或者毁灭}
	\label{ch:existence}
	\begin{motivation}
		莎士比亚笔下哈姆雷特的永恒困惑:"存在还是毁灭,这是一个问题。"\\
		《三体》中"生存是幸运的偶然"的冷峻陈述。\\
		\textbf{核心动机}:从文学和科幻的感性描述出发,建立对"生存"这一概念的物理与数学解释。
	\end{motivation}

	\section{从语言到自然:描述与现实的映射}
	当我们用语言描述世界时,我们实际上在进行一次信息压缩。语言将连续的自然现象离散化为可交流的符号,这个过程本身就蕴含着对"什么值得描述"的选择。

	例如,莎士比亚之所以写下"To be or not to be",是因为人类历史上充满了"生存"与"毁灭"的案例,使得这个概念在语言中被编码为一个高频词汇。语言的使用频率,反过来反映了自然现象的发生频率。

	\section{生存与毁灭:从幸运偶然到普遍现象}
	刘慈欣在《三体》中提出了一个令人不安的观点:"生存是幸运的偶然。"从宇宙尺度来看,这确实如此。在热力学熵增的大趋势下,任何有序结构的形成都需要极其特殊的条件——能量的持续输入、环境的稳定性、多个随机事件的巧合叠加。

	然而,如果我们缩小观察尺度,从微观到宏观,从个体到文明,"生存"与"毁灭"却变得普遍起来。一个细胞在培养皿中的存活,一个物种在生态系统中的延续,一个文明在历史长河中的兴衰,这些构成了我们日常经验的主要部分。

	\section{什么决定了生存与毁灭?}
	既然生存不是必然,那么是什么决定了一个系统能否存活?

	这里有两个关键原理:

	\subsection{人存原理}
	人存原理告诉我们:我们之所以能观察到现在这个宇宙,是因为我们存在于其中。换句话说,任何能够问出"为什么我存在"的观察者,都必须处于允许其存在的环境中。

	将这个原理推广到社会领域:任何能够持续存在的社会结构,必然满足了某些基本条件。如果某个结构毁灭了,那么它要么不满足这些条件,要么环境发生了剧烈变化。

	\subsection{中性选择原理}
中性选择原理指出,并非所有的变异都会被自然选择所筛选。很多变化对生存能力的影响微乎其微,它们会随基因漂变而保留或消失。

	在社会系统中,这意味着:很多文化、制度、行为模式并非因为"优越"而存在,仅仅是因为它们没有致命缺陷。它们能够在某些环境下存活,不等于它们在任何环境下都最优。
	
\end{document}