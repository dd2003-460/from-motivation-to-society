\documentclass[../main.tex]{subfiles}

\begin{document}
	
	\chapter{结构与效率:生存的物理学}
	\begin{motivation}
		为什么在这个混乱的宇宙中会出现有序的结构?\\
		\textbf{核心动机}:论证"结构"不是为了美,而是为了"效率"(在竞争中存活的概率)。
	\end{motivation}
	
	\section{自然选择作为过滤器}
	自然选择是宇宙中最基本的筛选机制。无论是物理结构、生物体还是社会组织,只有那些能够有效利用资源、维持自身稳定并传递信息的结构才能持续存在。这种筛选过程本质上是一种效率优化过程,它倾向于保留那些在特定环境中表现更优的结构。
	
	\section{结构是效率的物理固化}
	结构不是偶然形成的,而是效率需求的物理体现。在物理世界中,球形是最节能的形状;在生物界,分形结构优化了物质交换;在社会中,等级制度降低了协调成本。这些看似不同的现象背后,都遵循着同样的原则——效率最大化。
	
	\section{案例:从晶体到生物组织的必然性}
	\subsection{晶体结构的效率}
	晶体结构通过原子间的规则排列实现了能量的最低状态,这是一种自然形成的高效结构。类似的,社会中的组织结构也是为了实现某种效率而形成的。
	
	\subsection{生物组织的效率优化}
	生物体通过进化形成了高度优化的组织结构,这些结构在能量利用、信息传递和环境适应方面表现出卓越的效率。
	
\end{document}