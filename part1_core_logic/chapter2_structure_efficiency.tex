\documentclass[../main.tex]{subfiles}

\begin{document}
	
% ==========================================
% Chapter 2 · 结构与效率:生存的物理学
% ==========================================

\chapter{结构与效率:生存的物理学}
\label{chap:structure_efficiency}


\label{sec:selection_filter}

\section{自然选择作为过滤器}

\begin{motivation}
	在上一章,我们已经反复强调:存留并不等于最优。
	但如果自然选择并不追求最优,它究竟在系统中扮演什么角色?
	这一节的目标,是回答这个看似简单、却经常被误解的问题。
\end{motivation}

\subsection{不是选择最优,而是淘汰不可行}

\begin{readingnote}
	在绝大多数现实系统中,真正发生的并不是“选出最好的人”,
	而是“剔除明显不合格的人”。
	
	考试有及格线,产品有上线标准,申请有最低条件。
	超过这条线的人,往往会一起留下来;
	低于这条线的人,则直接被淘汰。
\end{readingnote}

这一现象揭示了一个常被忽略的事实:
自然选择的主要功能,并不是在所有可能方案中寻找最优解,
而是在持续运行中,排除那些\lowfreq{无法维持自身存在的结构}。

换句话说,自然选择更像一个过滤器,而不是排序算法。

\mathlink{math:selection_filter}{自然选择作为过滤过程的形式化描述}

\subsection{过滤阈值与及格线结构}

\begin{readingnote}
	如果一次考试的及格线是 60 分,
	那么 61 分和 90 分,在“是否通过”这一结果上,并没有本质差别。
	
	它们的差异,只会在下一轮筛选中才逐渐显现。
\end{readingnote}

将这一现象抽象化,我们可以得到一个稳定存在于多种系统中的结构:
\lowfreq{过滤阈值}。

系统并不关心个体有多优秀,
它只关心:是否低于某条不可逾越的生存边界。
一旦跌破这条边界,系统便会迅速将其排除。

\begin{logicmap}
	候选结构
	$\rightarrow$ 是否低于生存阈值
	$\rightarrow$ 淘汰 / 存留
	$\rightarrow$ 存留者之间不再强制排序
\end{logicmap}

这也解释了一个在直觉上容易被误解的现象:
在许多环境中,我们观察到的并不是“最优结构”,
而是一组\lowfreq{刚好没有被淘汰的结构}。

\mathlink{math:threshold_model}{过滤阈值与及格线模型}

\begin{readingnote}
	如果自然选择只是不断排除不可行方案,
	那么一个自然的问题是:
	
	既然系统并不追求最优,
	为什么最终留下来的结构,
	往往看起来却“相当高效”?
\end{readingnote}

这个问题,将在下一节中被系统回答:
效率并不是被追求的目标,
而是结构在约束下自然浮现的结果。


\section{结构为什么会产生效率}

\begin{motivation}
	在上一节中,我们已经把自然选择还原成一个非常“冷漠”的过程:
	它并不关心谁更优秀,只负责淘汰那些无法维持自身存在的结构。
	
	但这立刻引出了一个更直觉、更尖锐的问题:
	
	如果系统只是在不断排除不合格者,
	为什么最终留下来的东西,
	看起来往往\lowfreq{有层次、有组织,而且还挺高效}?
\end{motivation}

\subsection{为什么留下来的,往往是有层级的?}

\begin{readingnote}
	想一想你身边的组织形式。
	
	一个三五个人的临时群聊,
	可以没有分工、没有层级;
	但一旦人数变多、事情变复杂,
	你很快就会看到负责人、小组、流程的出现。
	
	并不是因为大家突然“喜欢官僚”,
	而是因为没有这些结构,事情根本运转不下去。
\end{readingnote}

这里有一个非常重要、但经常被忽略的事实:

\lowfreq{层级结构并不是为了提高上限,而是为了避免崩溃。}

在规模较小、任务简单的情况下,
扁平结构完全可以存活;
但当规模、复杂度或持续时间增加,
缺乏层级的结构更容易在协调、冲突或负担上直接失效。

自然选择并不会“偏好”层级结构,
但它会持续淘汰那些\lowfreq{在规模扩大后无法稳定运作的形态}。




	\begin{tcolorbox}[colback=yellow!5, title=\textit{读者行为预测}]
		
	读到这里,你可能会产生一个反感的念头:
	
	“这听起来像是在为复杂结构、甚至低效的官僚体系辩护。”
	
	这是一个非常自然的反应,
	但也是一个需要立刻澄清的误解。
\end{tcolorbox}

\begin{readingnote}
	\lowfreq{一个绕不开的问题:}
	
	如果层级结构只是为了“避免崩溃”,
	那是不是意味着——
	
	只要一个体系还能勉强运转,
	哪怕它臃肿、低效、令人痛苦,
	我们也应该对它保持容忍?
	
	换句话说:
	这种解释,会不会变成一种
	“反正还能活着,所以不用改”的借口?
\end{readingnote}

\begin{readingnote}
	\lowfreq{一个不太舒服,但很常见的例子:}
	
	想一想那些你明知效率很低、
	却长期存在的流程或制度。
	
	它们通常具备几个特征:
	\begin{itemize}
		\item 处理问题很慢,但不至于完全停摆
		\item 决策很保守,但很少立刻出大错
		\item 每一步都让人不满,但很少触发致命冲突
	\end{itemize}
	
	这些结构之所以还能存在,
	并不是因为它们“做得好”,
	而是因为它们\lowfreq{足够不容易出事}。
\end{readingnote}



这里的论断并不是:
层级结构一定更好,
而是:

\lowfreq{在某些规模和约束下,没有层级的结构更容易先死。}

自然选择只负责这一件事:谁先死。
至于留下来的结构是否优雅、高效、令人满意,
并不在它的考虑范围之内。

\subsection{效率不是被追求的目标,而是副产物}

\begin{readingnote}
	很多人在解释效率时,会下意识地引入“设计”:
	
	流程是被优化过的,
	结构是被精心规划的,
	所以结果才显得高效。
	
	但现实中,大量高效结构的起点,
	并不是“想提高效率”,
	而是“撑不下去了”。
\end{readingnote}

考虑一个极其常见的场景:

当事情还很少时,
随意沟通、随时调整,反而最省事;
但当任务变多、重复出现、持续时间拉长,
原本灵活的方式会迅速变成负担。

这时出现的结构变化——
分工、流程、固定接口——
并不是因为它们\emph{最优},
而是因为\lowfreq{它们让系统勉强活了下来}。

效率在这里,并不是目标,
而是一个结果:
\lowfreq{减少出错、减少冲突、减少崩溃机会}。

\begin{logicmap}
	规模扩大 / 持续时间增加
	$\rightarrow$ 原有方式更易失效
	$\rightarrow$ 结构被迫固定
	$\rightarrow$ 可重复性上升
	$\rightarrow$ 看起来更高效
\end{logicmap}

\subsection{为什么看起来低效的结构还能长期存在?}

\begin{readingnote}
	你可能马上会想到反例:
	
	很多组织、制度、流程,
	明明效率低下、令人不满,
	却依然可以存在多年,甚至几十年。
	
	如果效率是自然选择的结果,
	那它们为什么还没被淘汰?
\end{readingnote}

这个问题本身,恰恰说明了上一节结论的边界。

自然选择并不会比较两个都能活下来的结构,
只会淘汰\lowfreq{活不下来的那个}。

只要一个结构:
\begin{itemize}
	\item 不会立刻引发自身崩溃
	\item 能承受环境中的主要冲击
\end{itemize}

它就可以长期存在,
哪怕它在我们的主观评价中显得笨重、迟缓、浪费。

这也意味着:

\lowfreq{“还能活着”,本身就是一个极其宽松的标准。}

\mathlink{math:survival_constraint}{生存约束与结构存留的形式化条件}

\begin{readingnote}
	到这里,我们已经可以得到一个暂时但稳定的认识:
	
	结构与效率,
	并不是自然选择“追求”的目标,
	而是系统在反复避免失败过程中,
	逐渐留下来的形态特征。
	
	接下来,一个更具体的问题自然浮现:
	
	他是怎么在具体案例中展现出来的
\end{readingnote}

这个问题,将在下一节中通过具体案例展开:
王朝周期率,是因为什么


\section{效率的代价:结构僵化}
\label{sec:rigidity}

\section{案例:从晶体到生命}
\label{sec:cases}

\subsection{晶体结构的效率}
\label{subsec:crystal}

\subsection{生物组织的效率优化}
\label{subsec:biology}



\subsection{局部最优与路径依赖}
\label{subsec:local_optima}

\section{本章小结与衔接}
\label{sec:summary_ch2}

\subsection{回扣:结构是存在的结果}
\label{subsec:recap}

\subsection{衔接到下一章:语言如何固化结构}
\label{subsec:bridge_language}


\end{document}