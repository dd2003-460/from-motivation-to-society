\documentclass[../main.tex]{subfiles}

\begin{document}
	
	\chapter{语言与信息:分类的方法论}
	\begin{motivation}
		人类如何认知结构?我们不能处理无限的细节,必须进行"有损压缩"。\\
		本章是全书的数学核心,后续分析政治和经济都将基于这里的"变长编码"理论。
	\end{motivation}
	
	\section{语言即信息传递}
	语言不仅是交流的工具,更是认知的框架。通过语言,我们将复杂的现实世界简化为可处理的信息单元。这一过程本质上是一种信息压缩,即将高维的现实映射到低维的语言符号系统中。
	
	\section{结构的抽象表示:分类 (Classification)}
	分类是人类认知的基础能力,它允许我们将复杂的世界简化为可管理的概念类别。这种分类过程不仅是认知上的便利,更是生存的必需。
	
	\subsection{分类的本质:粗粒化 (Coarse-graining)}
	粗粒化是将精细的微观状态聚合成宏观状态的过程。在社会认知中,这意味着将具体的人和事归纳为一般性的概念和类型。这种处理方式虽然损失了细节信息,但大大提高了认知效率。
	
	\subsection{分类的代价:信息损失与误差分析}
	分类虽然提高了效率,但也带来了信息损失。这种损失可能导致刻板印象和偏见,因为我们在处理具体案例时可能会忽略其独特性。
	
	\section{编码理论:霍夫曼编码与社会分层}
	编码理论为理解社会现象提供了强有力的工具。特别是霍夫曼编码的思想——使用变长编码来优化传输效率——可以用来解释许多社会现象。
	
	\subsection{高频使用短编码(习惯/直觉)}
	类似于霍夫曼编码中频繁出现的字符使用较短的编码,社会生活中常见的行为模式和思维方式往往被压缩成简短的习惯或直觉反应。这提高了日常生活的效率,但也可能导致对复杂问题的简单化处理。
	
	\subsection{低频使用长编码(法律/逻辑)}
	对于不常见但重要的情况,社会发展出了复杂的规范体系,如法律条文和逻辑论证。这些"长编码"虽然复杂,但能够精确地处理特殊情况,避免因过度简化而导致的问题。
	
\end{document}