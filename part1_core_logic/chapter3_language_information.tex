\documentclass[../main.tex]{subfiles}

\begin{document}
	
	\chapter{语言:一种轻量级结构}
	
	\begin{motivation}
		说话是轻松的。
		写下一个句子,只需几秒。
		
		我们可以迅速构建复杂思想,
		快速组织概念,
		甚至模拟整个制度与系统。
		
		语言似乎是一种
		“几乎零成本”的结构工具。
		
		但语言本身,
		同样是长期选择与积累的结果。
	\end{motivation}
	
	\section{语言为何如此容易使用?}
	
	\subsection{低门槛的复杂结构}
	
	语言允许我们:
	
	\begin{itemize}
		\item 抽象概念
		\item 组织逻辑
		\item 构建规则
		\item 模拟未来
	\end{itemize}
	
	几乎不需要物理成本。
	
	与王朝相比,
	语言无需官僚体系;
	与生物相比,
	语言无需能量密集结构。
	
	\begin{readingnote}
		语言是一种低物理成本、
		高结构密度的系统。
	\end{readingnote}
	

	
	% =====================================================
	% 3.2 语言为何不是最优?
	% =====================================================
	
% =====================================================
% 3.2 语言为何不是最优?
% =====================================================

\section{语言为何不是最优?}

\subsection{我们习惯了它,却很少质疑它}

我们每天使用语言,
几乎从未停下来怀疑它。

但如果你认真观察,会发现:

\begin{itemize}
	\item 语法并不完全一致;
	\item 表达常常含糊;
	\item 同一个词在不同语境中意义不同。
\end{itemize}

语言远非“最优设计”。

\begin{questionbox}
	如果语言是一种表达工具,
	为什么没有演化成最简洁、最精准、最高效的系统?
\end{questionbox}

\begin{tcolorbox}[colback=yellow!5,title=\textit{读者行为预测}]
	你可能会回答:
	因为历史原因,或者因为人类不够理性。
	
	但请注意:
	即便今天我们知道如何优化,
	我们也几乎不会重构整个语言体系。
\end{tcolorbox}

这里出现了一个关键判断:

语言不需要最优,
只需要足够稳定。

\begin{readingnote}
	稳定性优先于局部最优,
	这是长期结构的一般特征。
\end{readingnote}

一旦一种语法被广泛接受,
改变它的成本远远高于保留它。

语言因此具有强烈的路径依赖。

% =====================================================
% 3.3 高频与压缩:看不见的长期压力
% =====================================================

\section{高频与压缩:一种看不见的压力}

\subsection{从一个日常问题开始}

为什么“的”“了”“是”如此短?

为什么法律条文却如此冗长?

这只是巧合吗?

\begin{tcolorbox}[colback=yellow!5,title=\textit{读者行为预测}]
	你可能会说:
	因为常用词需要方便。
	
	很好。现在我们把“方便”拆解开。
\end{tcolorbox}

\subsection{重复带来的压力}

一个表达如果每天被重复无数次,
它的成本会在长期累积。

如果某种表达方式可以被压缩,
那么高频表达首先会被压缩。

生活中也存在类似现象:

\begin{itemize}
	\item 公司将高频流程模板化;
	\item 手机把常用应用放在首页;
	\item 王朝把日常行政规则写得简单。
\end{itemize}

这并非聪明设计,
而是长期使用压力的结果。

\begin{readingnote}
	只要使用频率高度不均,
	且结构允许调整,
	压缩趋势几乎不可避免。
\end{readingnote}

\paragraph{附录说明}
形式化推导与编码理论说明见附录 A。

% =====================================================
% 3.4 反例与怀疑
% =====================================================

\section{这是否只是巧合?}

\subsection{看似的反例}

但你可能已经发现:

\begin{itemize}
	\item 某些高频词依然冗长;
	\item 某些低频词却很简短。
\end{itemize}

这是否推翻了前面的趋势?

\begin{tcolorbox}[colback=yellow!5,title=\textit{读者行为预测}]
	你可能会想:
	如果存在反例,
	趋势是否站不住?
\end{tcolorbox}

\subsection{趋势不是瞬间的聪明,而是被留下的轨迹}

趋势并不意味着,
每一个词此刻都是最合理的。

语言不是在某一天被重新设计,
也不是每一次使用都会被优化。

它更像是一条被反复踩出来的小路。

有些表达,
因为被不断使用,
慢慢变得顺手;
有些表达,
因为很少出现,
就保持着原样。

语言受到很多限制:

\begin{itemize}
	\item 发音必须说得出口;
	\item 旧规则不能随意废弃;
	\item 学习成本不能太高;
	\item 大多数人不会主动重构整个系统。
\end{itemize}

因此我们看到的,
不是“此刻最优”的语言,
而是“长期被保留下来的语言”。

\begin{tcolorbox}[colback=yellow!5,title=\textit{读者行为预测}]
	你可能会问:
	既然有更简洁的方式,
	为什么不直接替换?
\end{tcolorbox}

因为替换本身的成本,
往往高于维持现状。

高频表达,
在长期使用中被不断压缩;
低频表达,
则没有足够压力改变自己。

反例当然存在。

但自然选择从来不是消灭所有例外,
而是让多数情况沿着同一个方向缓慢移动。

\vspace{0.5em}

为了更清楚地看到这种压力,
我们可以看一个几乎没有历史惯性的系统:

输入法。

拼音输入法并不是自然演化,
而是工程设计。

但它的优化方向却非常一致:

\begin{itemize}
	\item 高频词优先联想;
	\item 高频组合自动补全;
	\item 常用短语只需极少按键。
\end{itemize}

甚至键盘布局本身,
也会根据使用频率调整快捷方式。

这里没有几千年的文化惯性,
没有发音结构的限制,
也没有教育制度的阻力。

只有一个目标:

减少高频操作的成本。

结果是什么?

高频内容被放得更近,
被缩得更短,
被调用得更快。

\begin{readingnote}
	当惯性被削弱,
	频率压力会更清晰地显现。
\end{readingnote}

语言之所以变化缓慢,
不是因为它不受这种压力,
而是因为它背负着历史。

但方向是一致的。

这不是一次设计,
而是一条被时间刻出来的轨迹。


% =====================================================
% 3.5 文学:压缩的极致
% =====================================================

\section{文学:压缩的极致形式}

当结构压缩到极致,
便产生了文学。

例如:
“存在还是毁灭,这是一个问题。”

短短几个词,
承载了数千年的困惑。

\begin{tcolorbox}[colback=yellow!5,title=\textit{读者行为预测}]
	你可能会误以为:
	文学创造了压缩。
	
	但恰恰相反。
\end{tcolorbox}

文学并没有创造压缩机制,
它只是利用了已经形成的高频结构。

高频概念,
被赋予简洁形式,
从而具备强烈的表达张力。

文学是结构长期收敛的结果,
不是原因。

% =====================================================
% 3.6 从语言到自然
% =====================================================

\section{从语言到自然}

我们已经得到三个判断:

\begin{enumerate}
	\item 语言结构可调整;
	\item 使用频率高度不均;
	\item 长期使用压力塑造结构。
\end{enumerate}

\begin{questionbox}
	如果自然本身也存在频率不均,
	是否也会形成类似的收敛结构?
\end{questionbox}

\begin{tcolorbox}[colback=yellow!5,title=\textit{读者行为预测}]
	你可能会怀疑:
	语言频率是否真的反映自然频率?
\end{tcolorbox}

这个怀疑是合理的。

我们需要从自然层面重新审视:

为什么某些结构能够持续存在,
而另一些迅速消亡?

% =====================================================
% 3.7 存在为何如此频繁?
% =====================================================

\section{存在为何如此频繁?}

我们每天谈论“存在”。

但我们很少问:

为什么“持续存在”的结构,
在自然中如此常见?

如果瞬时结构更常见,
语言是否会围绕“消散”构建?

\paragraph{逻辑衔接}

下一章(循环回到第一章)将进入自然层面:

\begin{itemize}
	\item 自然究竟是什么样的?
	\item 我们如何描述自然?

\end{itemize}

语言只是一个描述。

真正的问题在自然之中。

	
\end{document}