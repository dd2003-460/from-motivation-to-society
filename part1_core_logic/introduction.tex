\documentclass[../main.tex]{subfiles}

\begin{document}

	\chapter*{引言}
	\addcontentsline{toc}{chapter}{引言}

	\section{写作动机}

	一切始于一连串\textbf{——不连着实际上,因为是断续的灵感}——深夜的讨论。朋友们的质疑与思考像种子一样落地生根:如果社会科学真能像物理一样建立公理化体系,那么它的基石应该在哪里?是否存在一种统一的方法论,能够跨越经济学、政治学、文化人类学等学科的边界,给出一致的解释框架?

	反复思考之后,答案逐渐清晰:方法论本身,而非具体理论,才是传播和扩散的载体。一个清晰可推导的逻辑链条,远比一堆零散的案例更有生命力。因此,本书的写作动机不仅是为了回答"为什么",更是为了展示"如何问为什么"。

—————\textbar \textbf{
	\underline{没人问为啥吗}
	}
	
	那我来吧,因为\textbf{效率},这根后面信息论\textbf{(part1的第三章,按频率编码可以压缩信息)}也有关系
	
	\section{面向读者群体}

本书试图在两个极端之间寻找平衡:一方面是数理严谨性,另一方面是社科可读性。完全偏向任何一端都会损失另一端的读者——过于数学化的论述会吓退缺乏背景的读者,而过于感性的描述又会失去预测能力以及——\textbf{失去问为什么和得到数学原理解答的机会},简而言之就是失去{\large \emph{\textbf{问和答}}}的\textbf{\emph{机会}}(示范了一下信息压缩)。

	因此,本书的目标读者包括:
	\begin{itemize}
		\item 希望理解社会演化底层逻辑的自然科学背景者
		\item 愿意接受数学工具但不希望被公式淹没的社会科学研究者
		\item 对跨学科思考感兴趣的普通读者
	\end{itemize}

	\section{结构设计思路}

	\subsection{ 动机清晰,逻辑唯一}


本书坚持一个核心原则:每个章节必须从一个明确的
{\large \emph{\textbf{动机}}}出发,沿着{\large \emph{\textbf{唯一}}}的逻辑链条展开。动机必须清晰到可以用一句话概括,逻辑必须紧凑到每一步都不存在分岔。

	这种设计有两个目的:
	\begin{enumerate}
		\item \textbf{帮助泛化}:当逻辑路径唯一时,读者可以自然地思考"如果改变某个假设会怎样",从而学会举一反三
		\item \textbf{信息压缩}:清晰的逻辑路径本身就是一种高效的编码方式,使得复杂现象可以被压缩为少数核心原则
	\end{enumerate}

	\subsection{模块化章节与逻辑拓扑}

全书采用模块化设计,每个章节都是可以独立阅读的单元,同时又能与其他章节形成不同的逻辑拓扑结构:
	\begin{itemize}
		\item \textbf{树形结构}:从核心动机出发,逐层细化到具体案例
		\item \textbf{循环结构}:某些章节之间存在"衔尾蛇"式的递归关系(例如语言描述自然,自然反过来约束语言)
		\item \textbf{有向图}:更一般地,章节之间的依赖关系形成一个有向图,读者可以根据自己的兴趣选择不同的阅读路径
	\end{itemize}

	\section{如何阅读本书}

	\subsection{线性阅读路径}

如果这是你第一次接触这类跨学科思考,建议按照目录顺序阅读,这样能够完整地体验逻辑链条的展开过程。

	\subsection{跳跃阅读路径}

如果你已经具备相关知识背景,可以根据自己的兴趣选择章节。每个章节的\textbf{逻辑地图}都标明了该章节在整体框架中的位置和前后依赖关系,帮助你快速定位。

	\subsection{示例:为什么要问"为什么"}

	现在,让我们用一个具体问题来演示本书的核心方法论。

	假设有人问:"你为什么要这么认真地追问'为什么'?"

	回答:因为只有问"为什么",才能挖到现象的根本原因。

	紧接着的追问:"为什么要挖到根本原因?"

	回答:因为一旦掌握了根本原因,就不需要"集邮式"地收集各种具体案例。当遇到新问题或新现象时,你可以从根本原因出发,推导出可能的解决方案。即使做不到严格的演绎推理,至少也能大幅提高检索效率——你知道应该从哪个方向去寻找答案。

	再追问:"这有什么实际意义?"

	回答:这让我们有能力预言罕见现象和罕见问题。如果某个问题在历史上只出现过一次,通过分析其根本原因,我们可以判断类似情境下是否可能再次发生,以及如何应对。

	\subsection{章节预告}

上述思考路径将在后续章节中系统展开:
	\begin{itemize}
		\item \hyperref[ch:existence]{第1章 存在或者毁灭}:从存在本身的意义出发,讨论生存与毁灭的物理基础
		\item \hyperref[ch:structure]{第2章 结构与效率}:为什么有些结构能够存活,而有些必然崩溃?
		\item \hyperref[ch:language]{第3章 语言与分类}:我们如何用语言压缩信息,以及这种压缩本身如何限制我们的理解
	\end{itemize}

\end{document}
