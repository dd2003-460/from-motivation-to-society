\documentclass[../main.tex]{subfiles}

\begin{document}
	
	\chapter{生存策略的分化:大一统与碎片化}
	\begin{motivation}
		基于不同的地理输入,社会为了追求"效率"(Part I结论),必然演化出不同的顶层生存策略。
	\end{motivation}
	
	\section{中央集权:作为一种降低内部交易成本的解法}
	在某些地理环境下,如需要大规模水利设施的农业区,中央集权成为了一种效率最高的组织形式。通过集中决策和统一调配资源,中央集权体制能够有效降低大规模协作中的交易成本,实现资源的优化配置。
	
	\section{思想控制的必然性推导}
	在中央集权体制下,维持意识形态的一致性是确保系统稳定运行的重要手段。
	
	\subsection{逻辑推导:集权 $\to$ 统一调度 $\to$ 统一思想 $\to$ 抑制异端}
	从系统论的角度看,中央集权的治理体系为了确保政策的有效执行,需要实现统一调度。而统一调度的前提是统一思想,这就导致了对不同观点的抑制。这不是出于恶意,而是系统运行的结构性需要。
	
	\subsection{这并非恶意,而是系统维持"低熵态"的能耗最优解}
	从物理角度看,任何有序结构都需要持续消耗能量来对抗熵增。中央集权体制通过统一思想来降低系统的管理成本,这是维持系统稳定的一种能耗最小化的解决方案。
	
	\section{假设反证:如果不是地理环境?}
	如果我们假设文明差异的根源不是地理环境,而是其他因素,那么就无法解释为何在相似地理环境下会形成相似的社会结构。
	
	\subsection{基于条件概率的"架空推演"}
	通过比较不同地理环境下的文明发展轨迹,我们可以发现,即使在不同种族、文化背景下,相似的地理条件往往会孕育出相似的社会组织形式。这进一步证实了地理环境的基础性决定作用。
	
\end{document}