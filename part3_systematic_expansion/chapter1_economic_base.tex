\documentclass[../main.tex]{subfiles}

\begin{document}
	
	\begin{logicmap}
		\centering
		\begin{tikzpicture}[
			node distance=2.5cm, 
			auto, 
			block/.style={rectangle, draw=green!60!black, fill=green!5, thick, align=center, rounded corners},
			line/.style={-Latex, thick}
			]
			\node [block] (env) {环境 (Part II)};
			\node [block, right=of env] (soc) {社会系统};
			\node [block, above right=of soc] (mat) {物质 (经济)};
			\node [block, below right=of soc] (ideo) {意识 (文化/政治)};
			
			\draw [line] (env) -- (soc);
			\draw [line] (soc) -- (mat);
			\draw [line] (soc) -- (ideo);
		\end{tikzpicture}
		
		\vspace{1em}
		\textbf{当前位置}:环境确立了生存策略,社会系统随之分裂为"硬件"(经济)和"软件/操作系统"(文化与政治)。
	\end{logicmap}
	
	\chapter{物质分支:经济基础}
	\begin{motivation}
		这是社会存续的能量来源。应用Part I的"结构效率"理论,解释商业逻辑。
	\end{motivation}
	
	\section{供求关系与价值认同}
	经济活动本质上是一种价值交换过程,供求关系反映了市场参与者对价值的认知和认同。这种关系不仅受客观条件的影响,也受到主观认知的调节。通过供求机制,社会资源得以分配,价值得以体现。
	
	\section{不对称信息:商业利润的来源}
	在现实经济活动中,信息往往是不对称分布的。掌握更多信息的一方能够在交易中获得优势地位,这种信息优势构成了商业利润的重要来源。企业家精神在很大程度上就是发现和利用信息不对称的能力。
	
	\section{多金融产品:人为制造的"复杂编码"}
	金融市场发展出了各种复杂的金融工具,这些工具可以被视为经济系统中的"复杂编码"。就像在信息理论中使用变长编码来优化传输效率一样,金融市场通过创造多样化的金融产品来优化资源配置和风险分担。
	
\end{document}