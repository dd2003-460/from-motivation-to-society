\documentclass[../main.tex]{subfiles}

\begin{document}
	
	\chapter{意识分支 I:文化与民间制度}
	\begin{motivation}
		在硬性的法律之上,必须有软性的润滑剂。
		文化是利用"频率编码"压缩后的社会行为规范。
	\end{motivation}
	
	\section{文化作为一种算法:低成本的决策辅助}
	文化可以被视为一种社会算法,它为个体在面对复杂社会情境时提供了快速决策的捷径。这种算法通过长期的历史演化形成,凝结了无数代人的经验智慧。通过遵循文化规范,个体可以在不确定的环境中做出相对可靠的决策,而不必每次都重新评估所有选项。
	
	\section{道德与风俗:高频行为的短编码化}
	道德规范和风俗习惯相当于高频行为的短编码,它们将常见的社会互动模式固化为简洁的行为准则。正如在信息理论中高频字符使用短编码以提高传输效率一样,社会也将频繁出现的行为模式简化为道德准则,从而降低社会交往的成本。
	
	\section{模因传播:不问对错,只问强度}
	理查德·道金斯提出的"模因"概念为我们理解文化传播提供了有力工具。文化元素(模因)的传播并不完全取决于其真理性或合理性,而更多地取决于其传播强度和适应性。强势的文化模因往往具有易于理解和记忆、符合人类心理倾向、能够自我强化等特点。
	
\end{document}