\documentclass[../main.tex]{subfiles}

\begin{document}
	
	\chapter{意识分支 II:政治与权力结构}
	\begin{motivation}
		当民间自发秩序不足以应对熵增时,需要专门的"职能"来强行降熵。
		这就是政治的起源。
	\end{motivation}
	
	\section{权力的本质:对职能命令的认同}
	权力并不是一种实体,而是一种关系。它的本质在于人们对特定职能或角色的认同和服从。这种认同不是任意的,而是基于社会对秩序和效率的需求。权力的有效性取决于其是否能够成功地降低社会系统的熵值,即提高组织和协调的效率。
	
	\section{组织的抽象:从个人魅力到机构权威}
	早期的政治权力往往依附于个人魅力(克里斯玛型权威),但随着社会复杂度的增加,权力逐渐从个人转向机构。这种转变体现了社会对稳定性和可预测性的需求,是组织进化的必然结果。
	
	\section{博弈与制衡:权力系统的动力学}
	权力系统不是静态的,而是动态的博弈场域。不同利益集团之间的互动形成了复杂的制衡机制。这些制衡不仅是政治设计的结果,更是社会系统自我调节的体现,有助于防止权力过度集中导致的系统僵化。
	
\end{document}