\documentclass[../main.tex]{subfiles}

\begin{document}
	
	\chapter{制度的终极形态:法律}
	\begin{motivation}
		政治意志的最终固化。法律是社会系统中"定义最精确、编码最长"的指令集。
	\end{motivation}
	
	\section{法律的变长编码属性}
	法律条文的特点是详尽和精确,这正是其"变长编码"属性的体现。与道德规范和习俗等"短编码"不同,法律必须对复杂情况进行详细的规定,以减少解释上的歧义。这种详细的编码虽然增加了法律条文的长度,但提高了规范的精确度。
	
	\subsection{为什么法律条文必须冗长?(为了减少歧义/分类误差)}
	法律作为社会规范的最高形式,必须具有明确性和可操作性。因此,法律条文需要尽可能考虑到各种可能的情况,消除模糊性,从而减少在适用过程中的争议和误解。这种冗长性是法律精确性的保障。
	
	\section{责任的归因:基于条件概率的定责逻辑}
	法律责任的认定涉及复杂的因果关系分析。在确定责任归属时,需要考虑行为与结果之间的概率关系。
	
	\subsection{回顾Sbbm讨论:多链条因果的数学处理}
	在复杂的法律案件中,往往存在多条因果链条相互交织。这时需要运用概率论和统计学的方法,评估各种因果关系的可能性大小,从而合理分配责任。这种方法不仅体现了法律的严谨性,也体现了现代社会对复杂问题精细化处理的需要。
	
\end{document}