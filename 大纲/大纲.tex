\documentclass[openany]{book} % openany 允许章节从任意页开始,方便阅读

% =============================
% 宏包与设置
% =============================
\usepackage[a4paper, margin=1in]{geometry}
\usepackage{amsmath, amssymb, amsthm} % 数学公式与符号
\usepackage{graphicx}
\usepackage[colorlinks, linkcolor=blue, anchorcolor=blue, citecolor=green]{hyperref}
\usepackage{enumitem}
\usepackage{tikz} % 用于绘制逻辑拓扑图
\usetikzlibrary{shapes,arrows,positioning}
\usepackage{fancyhdr} % 页眉页脚
\usepackage{titlesec} % 标题格式
\usepackage{ctex} % 中文支持

% 字体设置 (可选,根据系统配置)
\setmainfont{Times New Roman}
% \setCJKmainfont{SimSun} 

% 定义定理与推导环境
\newtheorem{axiom}{公理}[chapter]
\newtheorem{theorem}{定理}[chapter]
\newtheorem{definition}{定义}[chapter]
\newtheorem{principle}{原理}[chapter]

% 自定义“推导与数学细节”环境,用于区分正文与硬核推导
\newenvironment{derivation}[1][数学推导与模型细节]{
	\par\vspace{1em}\noindent\textbf{\color{darkgray} [#1]} \itshape \small
	\begin{quote}\color{gray}
	}{
	\end{quote}\par\vspace{1em}
}

% =============================
% 书籍信息
% =============================
\title{\textbf{社会动力学与公理化体系}\\ \large ——从变长编码到文明演化的统一框架}
\author{D·declaim}
\date{\today}

\begin{document}
	
	% =============================
	% 封面与前言
	% =============================
	\maketitle
	
	\chapter*{前言:双路思想与写作动机}
	\addcontentsline{toc}{chapter}{前言}
	
	\section*{写作动机}
	本书源于与朋友(如sbbm, gcr, ctt等)的启发式讨论,旨在构建一套兼顾数理逻辑与社会科学直觉的统一方法论。
	\begin{itemize}
		\item \textbf{核心目标}:将直觉性的社会观察(政治、经济、文化)压缩为低熵的公理化逻辑(信息论、动力学)。
		\item \textbf{面向群体}:数理背景读者(寻找社会现象的底层方程)与社科背景读者(寻找理论的逻辑骨架)。
	\end{itemize}
	
	\section*{本书结构:逻辑拓扑}
	本书并非线性叙事,而是模块化的逻辑拓扑(树状分叉与圈状循环):
	\begin{itemize}
		\item \textbf{Part I 现实分支(Exploration Track)}:基于“架空游戏”视角的宏观推演,讨论历史的偶然与必然。
		\item \textbf{Part II 理论分支(Foundation Track)}:从信息压缩与动力学方程出发,推导社会结构的生成。
		\item \textbf{循环结构}:从最基本的动力出发,产生现象,最终结果又构成新的选择压(自指与自证预言)。
	\end{itemize}
	
	\tableofcontents
	\mainmatter
	
	% =============================
	% 第一部分:现实分支 (现象与直觉)
	% =============================
	\part{现实分支:演化游戏与历史拓扑 (Exploration Track)}
	\vspace{2em}
	\textit{本部分侧重于宏观视角的“架空模拟”,讨论在给定的宇宙与地球基础设定下,社会形态演化的偶然性与必然性。}
	
	\chapter{导论:起点的设定}
	\section{宇宙与地球的基础参数}
	\section{偶然与必然的比例:历史的条件概率}
	\begin{itemize}
		\item 讨论:如果重来一次,人类社会是否还会如此?
		\item 方法论:利用链条各环节的条件概率来统一“偶然”与“必然”。
	\end{itemize}
	\section{从生存压力到社会结构}
	
	\chapter{经济现实:供求与不完全信息}
	\section{供求关系的本质认同}
	\section{不对称信息与追涨杀跌}
	\section{商业逻辑的自然演化}
	\begin{derivation}[模型补充:商业策略的演化]
		讨论多金融产品如何作为“捕食策略”在信息不对称环境中演化。
	\end{derivation}
	
	\chapter{政治现实:权力的抽象与赋予}
	\section{权力即共识:对命令的认同}
	\section{职能的抽象化:从个人到组织}
	\section{利益交换:带有偏好的经济学}
	
	\chapter{文化现实:模因的传播强度}
	\section{情绪与煽动:不问方向,但求强度}
	\section{背景信息与“不完备”的定义}
	\begin{itemize}
		\item 案例:地图缺失与挠不到痒的同构性。
		\item 讨论:信息背景如何决定描述的有效性(基于gcr讨论)。
	\end{itemize}
	
	% =============================
	% 第二部分:公理化体系 (内核与推导)
	% =============================
	\part{公理化体系:结构、信息与动力学 (Foundation Track)}
	\vspace{2em}
	\textit{本部分为本书核心,从底层数学原理出发,自底向上推导社会科学的各个子领域。}
	
	\chapter{第一公理:信息与结构化}
	\label{chap:info_theory}
	\section{分类与粗粒化 (Coarse-graining)}
	\subsection{分类的无交性与完全性}
	\subsection{分类精细度与误差分析}
	\begin{derivation}[数学推导:分类单元大小的评估]
		利用比例误差与绝对误差定义分类的“分辨率”。
		$$ E = \sum |x_i - \hat{x}| \quad \text{vs} \quad E_{rel} = \frac{\Delta x}{x} $$
	\end{derivation}
	
	\section{信息压缩与编码}
	\subsection{有损压缩:语言与抽象}
	\subsection{无损压缩:频率派与变长编码}
	\begin{derivation}[推导:霍夫曼编码与社会分层]
		证明:高频事件(日常行为)对应短编码(低成本习惯),低频事件(变革)对应长编码。
		$$ L = \sum p_i \log_2(1/p_i) $$
		论述完全精细的分类等价于照搬元素(信息熵极大)。
	\end{derivation}
	
	\chapter{演化动力学:系统的数学描述}
	\section{状态空间的定义}
	\section{动力学方程:从牛顿到社会力}
	\subsection{微分方程与循环因果}
	\subsection{自指与自证预言 (Self-Reflexivity)}
	\begin{derivation}[模型:预期一致的激励]
		建立循环微分方程描述股票价格与权力稳固度:
		$$ \frac{dX}{dt} = f(X, \text{Expectation}(X)) $$
	\end{derivation}
	
	\section{势阱与稳态:拉格朗日视角}
	\subsection{社会势能与广义动量}
	\subsection{杆的摇晃:动量与稳定性的类比}
	\begin{derivation}[数学推导:哈密顿量在社会系统中的适用性]
		讨论在相对稳态环境下,如何构造社会系统的拉格朗日量 $\mathcal{L} = T - V$ 并分析其极值路径。
	\end{derivation}
	
	\chapter{介观统一论:模因、经济与政治的转换}
	\section{模因统一视角}
	\subsection{经济:压缩偏好的金钱模因}
	\subsection{政治:带有偏好的利益交换}
	\section{守恒律与网络性质}
	\subsection{经济网络的价值守恒}
	\subsection{政治交换的利益守恒}
	\subsection{文化传播的非守恒性(无压扩增)}
	\begin{derivation}[推导:守恒与耗散系统的区别]
		对比经济交易($A \to B$)与模因复制($A \to A+B$)的拓扑差异。
	\end{derivation}
	
	\chapter{因果与责任:条件概率的法律应用}
	\section{多链条因果分析}
	\section{条件概率与责任划归}
	\begin{derivation}[推导:基于条件概率的定责模型]
		基于与sbbm的讨论。计算在不施加某条件情况下,危害结果发生的概率变化:
		$$ P(Result | \neg Condition) \quad \text{vs} \quad P(Result | Condition) $$
		通过归一化计算各环节的责任权重。
	\end{derivation}
	
	\chapter{人存原理与现象归纳}
	\section{现象归纳的适用边界}
	\section{人存原理:幸存者偏差的物理化}
	\section{等效曲线拟合与噪声评估}
	
	% =============================
	% 附录:工具箱与补充材料
	% =============================
	\appendix
	\part{附录:工具与方法论索引}
	
	\chapter{数学工具详解}
	\section{信息论基础:从香农到变长编码}
	\section{动力系统稳定性分析}
	\section{复杂系统的概率评估方法}
	
	\chapter{灵感来源与致谢}
	\section{关于朋友们的启发性讨论 (sbbm, ctt, gcr等)}
	\section{推荐阅读与索引地图}
	
	\backmatter
	\bibliography{references}
	
\end{document}