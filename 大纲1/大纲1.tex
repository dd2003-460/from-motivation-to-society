\documentclass[openany, a4paper, 12pt]{book}

% ==========================================
% 1. 基础宏包配置
% ==========================================
\usepackage[UTF8]{ctex}     % 中文支持
\usepackage{geometry}       % 页面设置
\geometry{left=1in, right=1in, top=1in, bottom=1in}
\usepackage{fancyhdr}       % 页眉页脚
\usepackage{hyperref}       % 超链接目录
\hypersetup{colorlinks=true, linkcolor=blue, filecolor=magenta, urlcolor=cyan}

% ==========================================
% 2. 绘图与框图工具 (修复 TikZ 和 Environment undefined 错误)
% ==========================================
\usepackage{tikz}           % 绘图核心包
\usetikzlibrary{shapes.geometric, arrows.meta, positioning} % 载入TikZ扩展库

\usepackage{tcolorbox}      % 用于制作漂亮的文本框

% ==========================================
% 3. 自定义环境定义
% ==========================================

% 定义【逻辑地图】环境
% 使用 tcolorbox 包装,带有标题和边框
\newenvironment{logicmap}{
	\begin{center}
		\begin{tcolorbox}[
			colback=blue!5!white,      % 背景色:极淡蓝
			colframe=blue!50!black,    % 边框色:深蓝
			title=\textbf{逻辑地图 (Logic Map)}, % 标题
			fonttitle=\bfseries,
			sharp corners=south,       % 底部直角
			boxrule=0.5mm
			]
		}{
		\end{tcolorbox}
	\end{center}
}

% 定义【写作动机】环境
% 使用灰色背景强调
\newenvironment{motivation}{
	\vspace{0.5em}
	\begin{tcolorbox}[
		colback=gray!10,           % 背景色:淡灰
		colframe=gray!60,          % 边框色:深灰
		title=\textit{本章动机 (Motivation)},
		fonttitle=\bfseries,
		arc=0mm,                   % 直角边框
		leftrule=3mm               % 左侧加粗
		]
		\small \itshape % 字体变小并倾斜
	}{
	\end{tcolorbox}
	\vspace{1em}
}

% ==========================================
% 4. 书籍信息
% ==========================================
\title{\textbf{社会动力学与公理化体系}\\ \large ——从变长编码到文明演化的统一框架}
\author{D·declaim}
\date{\today}

% ==========================================
% 5. 正文开始
% ==========================================
\begin{document}
	
	\maketitle      % 生成封面
	\tableofcontents % 生成目录
	
	\mainmatter     % 正文页码开始
	
	% =================================================================
	% PART I: 核心动机与工具
	% =================================================================
	\part{核心逻辑:自然、结构与语言 (The Core Loop)}
	
	% 逻辑地图:这里使用了 tikz 绘图
	\begin{logicmap}
		\centering
		\begin{tikzpicture}[
			node distance=2.5cm, 
			auto, 
			block/.style={rectangle, draw=blue!60, fill=blue!5, thick, align=center, rounded corners},
			line/.style={-Latex, thick}
			]
			% 定义节点
			\node [block] (nature) {自然现象\\(熵增)};
			\node [block, right=of nature] (struct) {结构与效率\\(负熵)};
			\node [block, right=of struct] (lang) {语言与分类\\(压缩)};
			
			% 定义连线
			\draw [line] (nature) -- (struct);
			\draw [line] (struct) -- (lang);
			\draw [line, dashed] (lang) to[bend left=45] node[above, font=\footnotesize] {解释与重构} (nature);
		\end{tikzpicture}
		
		\vspace{1em}
		\textbf{当前位置}:我们正在打造解剖手术刀。在切入社会肌体之前,必须先理解“分类”和“描述”本身的物理意义。
	\end{logicmap}
	
	\chapter{自然现象:动力与混乱}
	\begin{motivation}
		如果要从最开始的地方研究,必须回到热力学。为什么宇宙不保持死寂?\\
		\textbf{核心动机}:确立“变化”是绝对的,而“稳定”是需要消耗能量维持的特例。
	\end{motivation}
	
	\section{最基本的动力:熵增与能量流}
	\section{次级产生的现象:耗散结构}
	
	\chapter{结构与效率:生存的物理学}
	\begin{motivation}
		为什么在这个混乱的宇宙中会出现有序的结构?\\
		\textbf{核心动机}:论证“结构”不是为了美,而是为了“效率”(在竞争中存活的概率)。
	\end{motivation}
	
	\section{自然选择作为过滤器}
	\section{结构是效率的物理固化}
	\section{案例:从晶体到生物组织的必然性}
	
	\chapter{语言与信息:分类的方法论}
	\begin{motivation}
		人类如何认知结构?我们不能处理无限的细节,必须进行“有损压缩”。\\
		本章是全书的数学核心,后续分析政治和经济都将基于这里的“变长编码”理论。
	\end{motivation}
	
	\section{语言即信息传递}
	\section{结构的抽象表示:分类 (Classification)}
	\subsection{分类的本质:粗粒化 (Coarse-graining)}
	\subsection{分类的代价:信息损失与误差分析}
	\section{编码理论:霍夫曼编码与社会分层}
	\subsection{高频使用短编码(习惯/直觉)}
	\subsection{低频使用长编码(法律/逻辑)}
	
	% =================================================================
	% PART II: 初始条件 - 地理环境决定论
	% =================================================================
	\part{初始输入:地理与历史的必然性 (The Geologics)}
	
	\begin{logicmap}
		\centering
		\begin{tikzpicture}[
			node distance=2cm, 
			auto, 
			block/.style={rectangle, draw=red!60, fill=red!5, thick, align=center, rounded corners},
			line/.style={-Latex, thick}
			]
			\node [block] (tool) {工具 (Part I)};
			\node [block, below=of tool] (geo) {地理环境\\(Input)};
			\node [block, right=of geo] (hist) {历史路径\\(Branching)};
			
			\draw [line] (tool) -- (geo) node[midway, right, font=\footnotesize] {应用分析};
			\draw [line] (geo) -- (hist);
		\end{tikzpicture}
		
		\vspace{1em}
		\textbf{当前位置}:工具已备好。现在我们将其应用于人类历史的唯一“硬输入”——地理环境,并观察由此引发的必然分支。
	\end{logicmap}
	
	\chapter{唯物史观的基石:地理决定论}
	\begin{motivation}
		为什么文明会有不同的形态?不要谈论虚无缥缈的“民族性”,那是结果不是原因。
		我们要寻找最初的那个“输入变量”。
	\end{motivation}
	
	\section{因果链条的回溯:史观的层级}
	\subsection{反驳文化决定论:文化是果不是因}
	\subsection{地理环境作为不可变更的初始条件}
	\section{案例分析:东西方大分流的地理根源}
	\subsection{治水与农业:中央集权的地理必然性}
	\subsection{破碎海岸与贸易:分权与契约的地理必然性}
	
	\chapter{生存策略的分化:大一统与碎片化}
	\begin{motivation}
		基于不同的地理输入,社会为了追求“效率”(Part I结论),必然演化出不同的顶层生存策略。
	\end{motivation}
	
	\section{中央集权:作为一种降低内部交易成本的解法}
	\section{思想控制的必然性推导}
	\subsection{逻辑推导:集权 $\to$ 统一调度 $\to$ 统一思想 $\to$ 抑制异端}
	\subsection{这并非恶意,而是系统维持“低熵态”的能耗最优解}
	\section{假设反证:如果不是地理环境?}
	\subsection{基于条件概率的“架空推演”}
	
	% =================================================================
	% PART III: 系统展开
	% =================================================================
	\part{系统展开:经济、文化与政治 (The Structure)}
	
	\begin{logicmap}
		\centering
		\begin{tikzpicture}[
			node distance=2.5cm, 
			auto, 
			block/.style={rectangle, draw=green!60!black, fill=green!5, thick, align=center, rounded corners},
			line/.style={-Latex, thick}
			]
			\node [block] (env) {环境 (Part II)};
			\node [block, right=of env] (soc) {社会系统};
			\node [block, above right=of soc] (mat) {物质 (经济)};
			\node [block, below right=of soc] (ideo) {意识 (文化/政治)};
			
			\draw [line] (env) -- (soc);
			\draw [line] (soc) -- (mat);
			\draw [line] (soc) -- (ideo);
		\end{tikzpicture}
		
		\vspace{1em}
		\textbf{当前位置}:环境确立了生存策略,社会系统随之分裂为“硬件”(经济)和“软件/操作系统”(文化与政治)。
	\end{logicmap}
	
	\chapter{物质分支:经济基础}
	\begin{motivation}
		这是社会存续的能量来源。应用Part I的“结构效率”理论,解释商业逻辑。
	\end{motivation}
	\section{供求关系与价值认同}
	\section{不对称信息:商业利润的来源}
	\section{多金融产品:人为制造的“复杂编码”}
	
	\chapter{意识分支 I:文化与民间制度}
	\begin{motivation}
		在硬性的法律之上,必须有软性的润滑剂。
		文化是利用“频率编码”压缩后的社会行为规范。
	\end{motivation}
	\section{文化作为一种算法:低成本的决策辅助}
	\section{道德与风俗:高频行为的短编码化}
	\section{模因传播:不问对错,只问强度}
	
	\chapter{意识分支 II:政治与权力结构}
	\begin{motivation}
		当民间自发秩序不足以应对熵增时,需要专门的“职能”来强行降熵。
		这就是政治的起源。
	\end{motivation}
	\section{权力的本质:对职能命令的认同}
	\section{组织的抽象:从个人魅力到机构权威}
	\section{博弈与制衡:权力系统的动力学}
	
	\chapter{制度的终极形态:法律}
	\begin{motivation}
		政治意志的最终固化。法律是社会系统中“定义最精确、编码最长”的指令集。
	\end{motivation}
	\section{法律的变长编码属性}
	\subsection{为什么法律条文必须冗长?(为了减少歧义/分类误差)}
	\section{责任的归因:基于条件概率的定责逻辑}
	\subsection{回顾Sbbm讨论:多链条因果的数学处理}
	
	% =================================================================
	% 结语与附录
	% =================================================================
	\backmatter
	
	\chapter{结语:循环的闭合}
	\begin{motivation}
		回到原点。我们构建的这个社会系统,最终又是如何反作用于“自然环境”的?
	\end{motivation}
	\section{人定胜天?社会系统对初始条件的修正}
	\section{新的循环:技术作为新的变量}
	
	\appendix
	\part{附录:数学工具箱与参考文献}
	\chapter{详细数学推导}
	\section{变长编码与熵的计算细节}
	\section{动力学方程的稳定性证明}
	
\end{document}